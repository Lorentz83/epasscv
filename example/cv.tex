\documentclass[a4paper]{epasscv} % the default language is english

%% you can specify the cv language as class parameter
%% you can require the black and white theme
% \documentclass[a4paper,italian,black]{epasscv}

%% or load with the following command, the babel optional parameter
%% is used to automatically load the babel package
% \cvloadlanguage[babel]{italian}
% \cvloadlanguage{english}
% \usepackage[english]{babel}

\cvname{Santa Claus}
\cvpicture{\includegraphics[height=3cm]{photo}}

\begin{document}

\begin{personalinfo}
  \cvaddress{An hidden town in North Pole}
  \begin{cvphones}
    \cvhome{+NP 123 456}\hfill
    \cvmobile{+NP 1234 567}
    
    \cvfax{+NP 999 888}\hfill
    \cvwork{+NP 999 888}
  \end{cvphones}
  \begin{cvinstantmessages}
    \cvskype{santa.claus} \hfill \cvhangout{santaclause@google.np}
  \end{cvinstantmessages}
  \cvemail{santaclaus@northpole.np}
  \begin{cvwebsites}
    \cvwebsite{http://www.northpole.np}
    \cvlinkedin{http://www.linkedin.com/in/santaclaus}
    \cvfacebook{http://www.facebook.com/santaclaus/}
    \cvgoogleplus{http://plus.google.com/santaclaus/}
    \cvwordpress{http://www.www.northpole.np/blog/}
    \cvgithub{https://github.com/santaclaus}
  \end{cvwebsites}
  \cvgender{\cvMale}\hfill
  \cvnationality{\cvEN}\\
  \cvdateofbirth{270-12-25}
\end{personalinfo}

\begin{cvpreferredjob}
  \only[en]{Bring gifts to the homes of the good children, preferably
    the night between December $24^{th}$ and $25^{th}$}%
  \only[it]{Portare i regali nelle case dei bambini buoni,
    preferibilmente la notte tra il 24 ed il 25 dicembre}%
\end{cvpreferredjob}

%% to support localization, a lot of commands accept an optional
%% parameter to hide their result if the cv is running for another
%% language
%
% \section[en]{You can write a custom section in English}
% \section[it]{Puoi scrivere una sezione personalizzata in Italiano}
%
% \only[en]{and complete with English text}
% \only[it]{e completarla con testo in italiano}

\section{\cvWork}

\begin{timespan}{270-07-01}{} % if a date miss it is replaced with "current"
  \begin{occupation}{Elves coordinator in Santa's secret village}
    Research and develop new toys, mainly in wood.
  \end{occupation}  
  \begin{company}[http://www.secretvillage.np]
    Santa's secret village -- North pole
  \end{company}
  \businessOrSector{Artisanship}
\end{timespan}


\section{cvEducation}

\begin{timespan}{1920-01-01}{1923-01-07}
  \begin{education}{Master in wooden carving}{ISCED 4}
    In this master I have learned how to carve wood, mainly to make
    wooden puppets.
  \end{education}
  
  \begin{institute}[http://www.geppetto.it/]
    Mr Geppetto store, Firenze -- Italy
  \end{institute}
\end{timespan}

\section{\cvPersonalSkills}

\cvmothertongue{\cvEN}
\begin{cvotherlanguages}
  \cvlanguage{\cvEN}{C1}{C1}{C1}{C1}{C1}
  \cvlanguage{any other}{C1}{C1}{C1}{C1}{C1}
\end{cvotherlanguages}


% \subsection{Driving licence(s)}
% A -- with an especial permit for flying reindeer

\section{Publications}

%% Yes, there is a special command for publications :)
\publication{Santa Claus and Mr. Geppetto}
  {How to carve puppets}
  {North Pole editions, 1883}

\subsection{\cvCommunicationSkills}
\subsection{\cvOrganisationalSkills}
\subsection{\cvJobSkills}
\subsection{\cvComputerSkills}

\section{\cvAdditionalInfo}
\section{\cvAnnexes}


%% In some country -- like Italy -- it is necessary to end the cv with
%% a standard closing that openly permit the organization to store and
%% manage the personal data written on the cv itself.
\cvclosing

\end{document} 
